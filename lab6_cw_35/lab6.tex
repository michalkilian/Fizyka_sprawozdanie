\documentclass[a4paper,10pt,twoside]{article}

\usepackage{amssymb}

\usepackage{amsthm}
\usepackage[polish]{babel}
\usepackage[utf8]{inputenc}
\usepackage[T1]{fontenc}
\usepackage{indentfirst}
\usepackage{caption}
\usepackage[top=2.5cm, bottom=2.5cm, left=2.5cm, right=2.5cm]{geometry}
\usepackage{graphicx}
\usepackage{makecell}
\usepackage{amsmath}
\usepackage{booktabs}
\usepackage{multirow}

\begin{document}
	
\section{}
da
\newpage
\section{title}
\newpage
\section{title}
\newpage
\section{Opracowanie wyników}
\noindent
Masa miedzi wydzielonej podczas elektrolizy na katodzie
$$m = 0,301g$$
Zmiana masy anod podczas elektrolizy
$$M = 0,298g$$
Oblicz wartość współczynnika elektrochemicznego wykorzystując wzór $k = \frac{m}{I\cdot t}$
$$k = \frac{0,301}{0,5*30*60}=0,000334[\frac{g}{C}]$$
Korzystając z otrzymanej wartości współczynnika $k$ oblicz, przy pomocy wzoru $F = \frac{\mu}{w\cdot k}$
$$F=\frac{63.58}{2*0,000334} = 95179,64[\frac{C/mol}]$$
Posługując się wyznaczoną doświadczalnie stałą Faradaya oblicz wielkość ładunku elementarnego
$$e = \frac{F}{N_a} = \frac{95179,64}{6,02*10^{23}} = 1.58*10^{-19}[C]$$

\paragraph*{OBLICZANIE NIEPEWNOŚCI POMIAROWEJ}
\noindent
\newline
$$m = 0,301g$$
Niepewność pomiaru masy miedzi wydzielonej podczas elektrolizy przyjmujemy jako\\
$$u(m) = 0,00058g$$
Oblicz niepewność wartości ładunku elektrycznego, który przepłynął przez elektrolit. W tym celu obliczamy niepewność pomiaru natężenia prądu wiedząc, że jest ona równa 
$$u(I) = (\text{klasa amperomierza * zakres / 100}) = 0,5 * 0,75 / 100 = 0,0038[A]$$
Oszacuj niepewność pomiaru czasu. W zależności od oceny wielkości tej niepewności można uwzględnić ją w dalszych obliczeniach albo pominąć ze względu na fakt, że jest zaniedbywalnie mała. 
\\W naszym przypadku $u(t) = 1s \hspace{10pt} \frac{u(t)}{t} = \frac{1}{1800} = 0,056\%$\\
Będziemy uwzględniać tę niepewność.

\vspace{5pt}
\noindent Ponieważ równoważnik elektrochemiczny miedzi obliczyliśmy ze wzoru $k = \frac{m}{I\cdot t}$ w którym występują tylko operacje mnożenia i dzielenia, złożona niepewność względna jest równa 
$$\frac{u(k)}{k} = \sqrt{\left[\frac{u(m)}{m}\right]^2 + \left[\frac{u(I)}{I}\right]^2 + \left[\frac{u(t)}{t}\right]^2} = \sqrt{\left[\frac{0,00058}{0,301}\right]^2 + \left[\frac{0,0038}{0,5}\right]^2 + \left[\frac{1}{1800}\right]^2} = 0,0079$$
$$u(k) = 0,0079 * 0,000334 = 2,6*10^{-6}[\frac{g}{C}] $$
Stała Faradaya oraz ładunek elementarny obliczane są z wzorów, w których obliczone k mnożone lub dzielone jest przez tablicowe stałe, których niepewności są pomijalnie małe więc 
$$
\frac{u(F)}{F} = \frac{u(k)}{k} \hspace{10pt} \frac{u(e)}{e} = \frac{u(k)}{k}
$$
$$u(f) = F\cdot\frac{u(k)}{k} = 95179,64*0,0079 = 751,92[\frac{C}{mol}]\hspace{10pt} u(e) = e\cdot\frac{u(k)}{k} = 1,58*10^{-19}*0,0079 = 1,2*10^{-21}[C]$$
\newpage
\paragraph*{Zestawienie wyników} Wszystkie wartości zostały zebrane w tabeli


\begin{table}[!htb]

\begin{tabular}{|c|c|c|c|c|c|}
	\hline
	&wartość tablicowa & wartość wyznaczona & różnica & niepewność & niepewność względna [\%] \\
	\hline
	k&0,0003294 &0,000334 & $5*10^{-6}$&$2,6*10^{-6}$&0,79\\
	\hline 
	F&96500&95179,64&1320,36&751,92&0,79\\
	\hline
	e&$1,6*10^{-19}$&$1,58*10^{-19}$&$2*10^{-21}$&$1,2*10^{-21}$&0,79\\
	\hline
	
\end{tabular}
\end{table}

\paragraph*{Porównanie mas blaszek}:\newline
Zmiana mas na anodach wyniosła 0,297g\\
Niepewność $u(m) = 0,00058g$\\
Sprawdzamy czy ubytek masy na anodach jest równy masie wytworzonej miedzi w granicach niepewności. Będzie to prawdą gdy $|y_1 - y_2| < U(y_1-y_2)$
$$|0,301-0,297| < 2*\sqrt{2*0,00058^2}$$
$$0,004 < 0,0016$$
Oznacza to, że ubytek masy nie jest równy masie wytworzonej miedzi w granicach niepewności.
\section{Wnioski}
Wyznaczona wartość równoważnika elektrochemicznego miedzi wynosi $0,000334 \pm 0,000005 [\frac{g}{C}]$,więc nie jest zgodna z wartością tabelaryczną w granicach niepewności. Zatem wyniku nie możemy uznać za poprawny.

Wyznaczona wartość stałej Faradaya wynosi $95179,64 \pm 751,92 [\frac{C}{mol}]$, więc nie jest zgodna z wartością tabelaryczną w granicach niepewności. Zatem wyniku nie możemy uznać za poprawny.

Wyznaczona wartość ładunku elementarnego wynosi $1,58\cdot10^{-19} \pm 1,2\cdot10^{-21}[C]$, więc nie jest zgodna z wartością tabelaryczną w granicach niepewności. Zatem wyniku nie możemy uznać za poprawny.

Otrzymane błędy mogą wynikać z konieczności obmycia miedzianych płytek, które wiązało się trzymaniem ich w dłoniach, co mogło spowodować starcie się miedzi z płytek.
\end{document}