\documentclass[a4paper,11pt]{article} % preambuła
\usepackage[polish]{babel}
\usepackage[utf8]{inputenc} % utf8, cp1250
\usepackage{pslatex}
\usepackage[T1]{fontenc}
\usepackage{times}
\usepackage{geometry}
\usepackage{algorithm}
\usepackage{graphicx}
\usepackage{xcolor}
\usepackage{wrapfig}
\usepackage{amsmath}
\usepackage{subcaption}
\usepackage{algpseudocode}
\usepackage{pgfplots}
\newgeometry{tmargin=1.5cm, bmargin=1.5cm, lmargin=1.5cm, rmargin=1.5cm}

\begin{document}
\section{Jakie są różnice w opisie przewodnictwa elektrycznego metali i
elektrolitów? Opisz, na czym polega proces elektrolizy.}
Charakterystyczną grupę przewodników prądu elektrycznego stanowią elektrolity. Są to przeważnie wodne roztwory zasad, kwasów i soli. Przy rozpuszczaniu kryształu wiązania jonowe pękają i atomy przechodzą do roztworu w postaci jonów poruszających się bezwładnie w roztworze. Jeśli przez roztwór ten przepuścimy prąd elektryczny, to ruch jonów staje się uporządkowany. Kationy zdążają do ujemnej elektrody, aniony do katody. Przepływowi prądu towarzyszy zobojętnianie jonów na elektrodach i wydzielanie się substancji na elektrodach. Proces ten nazywamy elektrolizą.

W metalach elektrony przewodnictwa stanowią elektrony walencyjne poszczególnych atomów. W sieci krystalicznej odrywają się one od swoich atomów i zaczynają swobodnie poruszać się w całej objętości metalu, tworząc tzw. gaz elektronowy. Koncentracja elektronów przewodnictwa w metalach nie zależy od temperatury.

Elektrolity, podobnie jak metale, przewodzą prąd elektryczny. Jednak mechanizm tego zjawiska w elektrolitach jest inny niż w metalach. W elektrolitach nośnikiem ładunku są jony obu znaków. Oznacza to przede wszystkim przemieszczanie się mas dziesiątki i setki tysięcy razy przewyższających masę elektronu. Natężenie, przewodność właściwa roztworu zależą od wielkości jonów. 

Oporność metali pod wpływem ogrzewania rośnie, a elektrolitów maleje. 
\section{Podaj prawa elektrolizy Faradaya.}
I prawo:\\
Masa substancji wydzielonej podczas elektrolizy jest proporcjonalna do ładunku, który przepłynął przez elektrolit. 
$$m = qk=Itk$$
gdzie: \\
$k$ - równoważnik elektrochemiczny\\
$q$ - ładunek elektryczny\\
$l$ - natężenie prądu elektrycznego\\
$t$ - czas.

II prawo:\\
Ładunek $q$ potrzebny do wydzielenia lub wchłonięcia masy m jest dany zależnością
$$q = \frac{Fmz}{M}$$
gdzie:\\
$F$ - stała Faradaya\\
$z$ - ładunek jonu\\
$M$ - masa molowa jonu.

\section{Jaką masę substancji wydzieli podczas przepływu przez elektrolit prąd
o natężeniu 1 ampera w czasie jednej sekundy? Podaj nazwę tej wielkości}
Masa substancji wytworzona podczas przepływu przez elektrolit prądu o natężeniu 1 ampera w czasie jednej sekundy wynosi k, gdzie k to równoważnik elektrochemiczny.
Obliczymy to korzystając z pierwszego prawa Faradaya
$$ k- nieznane, I = 1A, t = 1s$$
$$ m = kIt = k \cdot 1 \cdot 1 [g] = k [g] $$
Jednostką równoważnika elektrochemicznego jest $\frac{kg}{C}$ - kilogram przez kulomb.

Równoważnik chemiczny to wartość stosowana w elektrochemii równa masie substancji wydzielonej przy przepływie przez elektrolit ładunku elektrycznego 1C.
\section{Wyjaśnij na przykładzie pojęcia: masa molowa, wartościowość, kation,
anion, katoda, anoda.}
Masa molowa - masa jednego mola substancji chemicznej. Jednostką masy molowej jest $\frac{kg}{mol}$.\\
Dla przykładu obliczmy masę molową związku $H_2SO_4$. 
$$m_{H_2SO_4} = 2 \cdot 1 + 32 + 4*16 = 98 \frac{g}{mol}$$

Wartościowość - cecha pierwiastków chemicznych oraz jonów określająca liczbę wiązań chemicznych, którymi dany pierwiastek lub jon może łączyć się z innymi.
Dla przykładu chcąc obliczyć wartościowość siarki związku $H_2SO_4$
$$H=1, O=-2\text{, więc, aby wyszło zero:} 2\cdot 1 + S + 4\cdot -2 = 0$$
$$ 2 + S - 8 = 0 \rightarrow S = 6 $$

Kation - jon o ładunku dodatnim, w którym występuje nadmiar protonów w stosunku do elektronów. Podczas elektrolizy kation podąża do elektrody ujemnej, zwanej katodą.

Anion - jon o ładunku ujemnym. Podczas elektrolizy anion podąża do elektrody dodatniej, zwanej anodą.

Katoda - elektroda, przez którą z urządzenia wypływa prąd elektryczny. W odbiornikach prądu elektrycznego katoda jest elektrodą ujemną, a w źródłach prądu - dodatnią.

Anoda - elektroda, przez którą prąd elektryczny wpływa do urządzenia. W odbiornikach prądu elektrycznego anoda jest elektrodą dodatnią, a w źródłach prądu - ujemną.

\section{Zdefiniuj pojęcia: 1 amper, 1 wolt i 1 kulomb. Wyraź te jednostki za
pomocą jednostek podstawowych układu SI.}

Amper - jednostka natężenia prądu elektrycznego. Jest podstawową jednostką układu SI. Najczęściej oznaczana symbolem A.\\
1 amper to niezmieniający się prąd elektryczny, który płynąc ww dwóch równoległych, prostoliniowych, nieskończenie długich przewodach o znikomo małym przekroju kołowym, umieszczonych w próżni w odległości 1m od siebie, spowodowałby wzajemne oddziaływanie przewodów na siebie z siłą równą $2\cdot 10^{-7} N$ na każdy metr długości przewodu.

Wolt - jednostka potencjału elektrycznego, napięcia elektrycznego i siły elektromotorycznej. W jednostkach SI: 
$$1V = \frac{1kg\cdot m^2}{1A\cdot s^2}$$
Najczęściej oznaczany symbolem $V$. \\
1 wolt jest różnicą potencjałów elektrycznych pomiędzy dwoma punktami przewodu liniowego, w którym płynie niezmieniający się prąd o natężeniu jednego ampera, zaś moc pobierana pomiędzy tymi punktami jest równa jednemu watowi.

Kulomb - jednostka ładunku elektrycznego. W jednostkach SI:
$$1C = 1A\cdot s$$
Najczęściej oznaczany symbolem $C$.\\
1 kulomb to ładunek elektryczny przenoszony w czasie 1 sekundy przez prąd o natężeniu wynoszącym 1 amper.

\section{W jaki sposób (szeregowo czy równolegle) należy włączyć amperomierz do
obwodu? Dlaczego?}
Amperomierz należy włączyć do obwodu szeregowo, ponieważ mierzy natężenie prądu płynącego w obwodzie. Jego oporność wewnętrzna jest bardzo mała, więc spadek napięcia na nim jest minimalny. Jeśli podłączylibyśmy amperomierz równolegle spowodowałby zwarcie w tym obwodzie.

\section{Ile atomów miedzi osadzi się na elektrodzie po przepłynięciu przez
elektrolit ładunku elektrycznego równego stałej Faradaya?}
Stała Faradaya : $9,648533289 * 10^4 \frac{C}{mol}$
$$m = 0,3294 * 9,648533289 * 10^4 = 3,1782268653966 * 10^4 = 31782,27mg = 31,78g/mol$$
$$1u - 1,66 * 10^{-24}$$
$$63,546 u - 105,486 * 10^{-24} g$$

$$1 \text{ atom miedzi} - 105,486 * 10^{-24} g $$
$$x \text{ atomów miedzi} - 31,78 g $$


Zatem na elektrodzie osadzi się $3.0127 * 10^{23}$ atomów miedzi.
\section{Ładunek elektryczny Q jest iloczynem natężenia prądu I oraz czasu t:
Q = I t. Korzystając z prawa przenoszenia niepewności oszacuj niepewność
wyznaczenia ładunku z pomiarów I i t. }

$I$ oraz $t$ - zmierzone.\\
Znając wartości niepewności pomiaru $I - u(I)$ oraz $t - u(t)$ możemy obliczyć niepewność wyznaczania ładunku z pomiarów:
$$u_c(Q) = \sqrt{\sum_k [\frac{dy}{dx}u(x_k)]^2} = \sqrt{\frac{d[Q(I,t)]}{dt} + \frac{d[Q(I,t)]}{dI}} = \sqrt{[t*u(I)]^2 + [I*d(t)]^2}$$

\end{document}