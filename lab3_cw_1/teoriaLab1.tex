\documentclass[a4paper,11pt]{article} % preambuła
\usepackage[polish]{babel}
\usepackage[utf8]{inputenc} % utf8, cp1250
\usepackage{pslatex}
\usepackage[T1]{fontenc}
\usepackage{times}
\usepackage{geometry}
\usepackage{algorithm}
\usepackage{graphicx}
\usepackage{xcolor}
\usepackage{wrapfig}
\usepackage{amsmath}
\usepackage{subcaption}
\usepackage{algpseudocode}
\usepackage{pgfplots}
\newgeometry{tmargin=1.5cm, bmargin=1.5cm, lmargin=1.5cm, rmargin=1.5cm}

\begin{document}
\section{ Definicje i podstawowe zależności dla wielkości kinetycznych opisujących
ruch obrotowy (kąt, prędkość kątowa, przyspieszenie kątowe, jednostajny i
niejednostajny ruch obrotowy){\color{red}{Gotowe}}}
Kąt - przestrzeń zawarta między dwiema półprostymi wychodzącymi z tego samego punktu.\\
Prędkość kątowa - wielkość wektorowa opisująca ruch obrotowy. Jest ilorazem kąta $\alpha$ zakreślonego przez promień wodzący punktu poruszającego się po okręgu do czasu $t$, w którym ten kąt został zakreślony. $\omega = \frac{\alpha}{t}$. Jednostką prędkości kątowej jest $[\omega] = \frac{rad}{s}$.\\
Przyspieszenie kątowe - pochodna prędkości kątowej po czasie. $\vec{a_{r}} = \frac{d \vec{\omega}}{dt}.$\\
Ruch jednostajny obrotowy - ruch, w którym wektor prędkości kątowej $\vec{\omega}$ ma stałą wartość, kierunek i zwrot. Kierunek jest równoległy do bryły. \\
Ruch niejednostajny obrotowy - ruch po okręgu ze zmienną wartością prędkości kątowej.


\section{Definicje i podstawowe zależności dla wielkości dynamicznych opisujących
ruch obrotowy (moment bezwładności, momentu pędu, moment siły, druga
zasada dynamiki dla ruchu obrotowego).{\color{red}{Gotowe}} }
Moment bezwładności - opisuje sposób rozkładu masy wokół osi obrotu. Jest funkcją kwadratu odległości elementów masy od osi obrotu. $I = mR^2$ . $ [I] = kg \cdot m^2$\\
Pęd - iloczyn masy $m$ oraz prędkości $v$ punktu. $\vec{p} = m \vec{v}$\\
Moment pędu - iloczyn wektorowy promienia wodzącego $R$ elementu masy i wektora pędu tego elementu. $ L = r X p$ Dla ciała o momencie bezwładności $I$ obracającego się wokół ustalonej osi z prędkością kątową $\omega$ moment pędu można wyrazić wzorem: $L = I \omega$.\\
Moment siły - wielkość wektorowa równa iloczynowi wektorów ramienia siły i siły. $ \vec{M} = \vec{r} X \vec{F} = r F sin(\alpha)$, gdzie $\alpha$ - kąt między wektorem ramienia siły i wektorem siły. Jednostką jest niutonometr $[M] = 1 Nm$ \\
Druga zasada dynamiki dla ruchu obrotowego - jeżeli wypadkowy moment siły $M$ działający na ciało jest różny od zera, to ciało porusza się z przyspieszeniem kątowym $\epsilon$. Przyspieszenie kątowe bryły jest wprost proporcjonalne do wypadkowego momentu siły i odwrotnie proporcjonalne do momentu bezwładności $I$ : $\vec{\epsilon} = \frac{\vec{M}}{I}$.


\section{Definicja momentu bezwładności. Wyprowadzenie momentu bezwładności
dla jednorodnego pręta o długości l i masie m względem osi prostopadłej
do pręta i przechodzącej przez jego środek masy.}

Moment bezwładności punktu materialnego jest definiowany
jako iloczyn masy i kwadratu odległości od osi obrotu. Momenty bezwładności brył
sztywnych, tak I0 jak i IS , wyraża się jako całkę oznaczoną\\
$I = \int_m r^2 dm$\\
gdzie r jest odległością elementu masy dm od osi obrotu. Całkę można analitycznie
obliczyć dla brył jednorodnych o prostych kształtach.
\\
Dla jednorodnego pręta o długości l i masie m względem prostopadłej osi przechodzącej przez środek masy mamy:\\
\begin{center}
gęstość liniowa $ \lambda = \frac{m}{l} dm = \lambda dx $
\end{center}
$$ I = \int_{\frac{-l}{2}}^{\frac{l}{2}} \lambda x^2 dx = \frac{\lambda x^3}{3}|_{\frac{-l}{2}}^{\frac{l}{2}} = \frac{\lambda l^3}{24} + \frac{\lambda l^3}{24} = \frac{ml^2}{12}$$

\section{Twierdzenie Steinera dla momentu bezwładności i przykłady jego
zastosowania.}
Twierdzenie Steinera: Moment bezwładności bryły sztywnej względem dowolnej osi jest równy sumie momentu bezwładności względem osi równoległej do danej i przechodzącej przez środek masy bryły oraz iloczynu masy bryły i kwadratu odległości między tymi dwiema osiami, co wyraża się wzorem: $I = I_0 + md^2$, gdzie: \\
$I_0$ - moment bezwładności względem osi przechodzącej przez środek masy\\
$I$ - moment bezwładności względem osi równoległej do pierwszej osi\\
$d$ - odległość między osiami\\
$m$ - masa bryły.\\

Przykładowo obliczmy moment bezwładności pręta o długości l i masie m dla osi przechodzącej przez koniec pręta.

Z twierdzenia Steinera $$I = I_0 + md^2 = \frac{ml^2}{12} + m (\frac{l}{2})^2 =  \frac{ml^2}{12} +  \frac{ml^2}{4} =  \frac{ml^2}{3}$$

porównajmy otrzymany wynik z bezwładnością otrzymaną z całki 
$$ I = \int_{0}^{l} \lambda x^2 dx = \frac{\lambda x^3}{3}|_{0}^{l} = \frac{\lambda l^3}{3} - 0 = \frac{ml^2}{3}$$

\section{ Ruch harmoniczny, równanie ruchu i parametry opisujące ruch (amplituda,
okres, częstość, częstotliwość)}
Amplituda - największe wychylenie z położenia równowagi w ruchu drgającym lub falowym. \\
Okres - czas wykonania jednego pełnego drgania w ruchu drgającym. Dla fali oznacza to odcinek czasu między dwoma punktami fali o tej samej fazie. $ T = \frac{1}{f}$, gdzie $f$ to częstotliwość.\\
Częstość - (w fizyce ruchu drgającego lub falowego) wielkość określająca, jak szybko powtarza się dane zjawisko okresowe. Powiązana jest z częstotliwością i okresem.\\
Częstotliwość - wielkość fizyczna określająca liczbę cykli zjawiska okresowego występujących w jednostce czasu. $f = \frac{n}{t}$, gdzie: \\
$f$ - częstotliwość\\
$n$ - liczba drgań\\
$t$ - czas, w którym te drgania występują.\\

Ruch harmoniczny - ruch drgający, w którym na ciało działa siła o wartości proporcjonalnej do wychylenia ciała z położenia równowagi, skierowana zawsze w stronę punktu równowagi. Ruch, który powtarza się w regularnych odstępach czasu np. drgania ciała zawieszonego na sprężynie, wahadło.

Zależność przemieszczenia x(t) ciała w ruchu harmonicznym opisuje wzór
$$x(t) = A \cos (\omega t + \phi)$$

\section{Wahadło matematyczne. Opis ruchu wahadła matematycznego dla małych
drgań. Okres drgań tego wahadła.{\color{red}{Gotowe}}}

Wahadło matematyczne to punkt materialny poruszający się po okręgu w płaszczyźnie pionowej w jednorodnym polu grawitacyjnym.
W przybliżeniu małych kątów ($\phi \le 5^o$), tj. drgań o małej amplitudzie, częstość kołową $\omega$ oraz okres $T$ drgań wahadła matematycznego możemy wyznaczyć korzystając z poniższych wzorów:\\
$ \omega = \sqrt{\frac{mgL}{I}} $ oraz $ T = 2 \pi \sqrt{\frac{I}{mgL}}$, gdzie:\\
m - masa ciężarka\\
g - przyspieszenie ziemskie\\
L - odległość dzieląca punkt zawieszenia wahadła od środka jego masy równa długości linki\\
I - moment bezwładności wahadła względem jego punktu zawieszenia.

Okres drgań wahadła matematycznego: $T = 2\pi \sqrt{\frac{L}{g}}$

\section{Wahadło fizyczne. Przybliżony opis ruchu wahadła fizycznego za pomocą
równania ruchu harmonicznego. Okres drgań wahadła fizycznego w
przybliżeniu harmonicznym.{\color{red}{Gotowe}}}
Wahadłem fizycznym nazywamy bryłę sztywną mogącą obracać się wokół osi obrotu $O$ nie przechodzącej przez środek masy $S$. Dla wahadła fizycznego moment siły powstaje pod wpływem siły ciężkości. Dla wychylenia $\theta$ jest równy $M = mga\sin(\theta)$, gdzie $a$ oznacza odległość środka masy $S$ od osi obrotu $O$. 

Równanie ruchu wahadła można zapisać jako $I_0 \frac{d^2\theta}{dt^2} = -mga\sin(\theta)$, gdzie $I_0$ jest momentem bezwładności względem osi obrotu przechodzącej przez punkt zawieszenia $O$. Jeżeli ograniczyć ruch do małych kątów wychylenia, to sinus kąta można zastąpić samym kątem w mierze łukowej, czyli $\sin(\theta) \approx \theta$. Wtedy: \\
$\frac{d^2\theta}{dt^2} + \omega_0^2 \theta(t) = 0$, gdzie $\omega_0^2 = \frac{mga}{I_0}$.\\ Jest to równanie oscylatora harmonicznego, którego rozwiązanie : \\
$ \theta = \theta_m \cos(\omega_o t + \alpha)$ przedstawia ruch harmoniczny. Amplituda $\theta_m$ i faza $\alpha$ zależą od warunków początkowych. \\
Okres drgań T wynosi : $ T = 2 \pi \sqrt{\frac{I_0}{mga}}$
\end{document}