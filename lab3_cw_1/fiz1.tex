\documentclass[a4paper,10pt,twoside]{article}
%\usepackage{amssymb}
%\usepackage{amsthm}
\usepackage[polish]{babel}
\usepackage[utf8]{inputenc}
\usepackage[T1]{fontenc}
\usepackage{indentfirst}
\usepackage{caption}
\usepackage[top=2.5cm, bottom=2.5cm, left=2.5cm, right=2.5cm]{geometry}
\usepackage{graphicx}
\usepackage{makecell}
\usepackage{amsmath}
\usepackage{booktabs}
\usepackage{multirow}


\begin{document}
	
	\begin{center}
		\bgroup
		\def\arraystretch{1.5}
		\begin{tabular}{|c|c|c|c|c|c|}
			\hline
			EAIiIB & \multicolumn{2}{|c|}{Michał Kilian} & Rok II & \multicolumn{2}{|c|}{Grupa 5a} \\
			\hline
			\multicolumn{3}{|c|}{\begin{tabular}{c}Temat: \\ Wahadło proste \end{tabular}} & 
			\multicolumn{3}{|c|}{\begin{tabular}{c}Numer ćwiczenia: \\ 0 \end{tabular}} \\
			\hline
			\begin{tabular}{@{}c@{}}Data wykonania\\10.10.2018r.\end{tabular} & \begin{tabular}{@{}c@{}}Data oddania\\12.10.2018r.\end{tabular} & 
			\begin{tabular}{c}Zwrot do\\poprawki\\\phantom{data} \end{tabular} & \begin{tabular}{c}Data oddania\\\phantom{data}\end{tabular} &
			\begin{tabular}{@{}c@{}}Data zaliczenia\\\phantom{data}\end{tabular} & \begin{tabular}{c}Ocena\\\phantom{ocena}\end{tabular} \\[4ex]
			\hline
		\end{tabular}
		\egroup
	\end{center}
	
	\section{Cel ćwiczenia}
	Zaznajomienie się
	z typowymi metodami opracowania danych pomiarowych
	przy wykorzystaniu wyników pomiarów dla wahadła pro
	stego 
	

	Wahadło matematyczne to punktowa masa $m$ zawieszona na nieważkiej i nierozciągliwej lince poruszająca w jednorodnym polu grawitacyjnym.
	W doświadczeniu wykorzystamy bardzo dobre przybliżenie takiego układu jakim jest ciężka metalowa kulka zawieszona na nitce.
	
	Aby znacząco uprościć obliczenia przyjmiemy $\sin\theta\approx\theta$ co jest prawdą dla małych wartości kąta $\theta$ zgodnie z
	twierdzeniem Taylora. Dzięki temu ograniczamy wpływ oporu powietza na wyniki, a z uproszczonego równania ruchu wahadła
	uzyskujemy następujacą zależność
	\begin{equation}
	T=2\pi\sqrt{\frac{l}{g}}
	\end{equation}
	gdzie $T$ - okres drgań, $l$ - długość nici, $g$ - przyspieszenie grawitacyjne. Po przekształceniu otrzymujemy wzór roboczy pozwalający
	na wyznaczenie wartości przyspieszenia grawitacyjnego dla Ziemi
	\begin{equation}
	\label{eq:working_g}
	g=\frac{4\pi^2l}{T^2}
	\end{equation}
	\newpage
	\section{Wykonanie ćwiczenia}
	
	\begin{enumerate}
		\item Zapoznać się z budową mikroskopu
		\item Na obu powierzchniach płytki zrobić kreski, jedna nad drugą cienkim pisakiem (ewentualnie wykorzystać istniejące kreski)
		\item Zmierzyć śrubą
		mikrometryczną
		grubość płytki d w pobliżu kresek. 
		\item Ustaw  badaną
		płytkę
		na  stoliku  mikroskopu  w  uchwycie  i  dobierz  ostrość
		tak  by  uzyskać
		kontrastowy obraz. Regulując położenie stolika pokrętłem 7a zaobserwuj górny i dolny 
		ślad zaznaczony na płytce. 
		\item Pokrętłem 7b przesuń stolik mikroskopu do momentu uzyskania ostrego obrazu śladu na górnej powierzchni płytki.
		\item Odczytaj położenie $a_g$ wskazówki czujnika mikrometrycznego. 
		\item Przesuń stolik mikroskopu do położenia, w którym widoczny jest ślad na dolnej powierzchni płytki (pokrętłem 7b). 
		\item Ponownie odczytaj położenie $a_d$ wskazóWki czujnika.
		\item Odczyty zanotuj w tabeli 1, 2 lub 3.
	\end{enumerate}
	
	
	\section{Wyniki pomiarów}
	\paragraph{Obliczenie grubości rzeczywistej dla płytki szklanej} WSTAWIĆ OBLICZENIA
\\	\begin{table}[!htbp]
		\caption{\textbf{}}
		\centering
		\def\arraystretch{1.4}
		\begin{tabular}{|c|c|c|c|}
			\hline
			\multicolumn{4}{|l|}{\makecell{materiał: szkło\hspace{115pt} \\ grubość rzeczywista d = WSTAWIĆ[mm] \\ niepewność \hspace{55pt}$u(d) = 0,01$[mm]\hspace{20pt}}} 
			\\ \hline
			\multirow{2}{*}{lp.} & \multicolumn{2}{c|}{Wskazanie czujnika} & \makecell{grubość \\pozorna } \\
			\cline{2-4}
			  
			\multirow{2}{*}{} & $a_d$[mm] & $a_g$[mm] & $h = a_d - a_g$[mm] \\ \hline
			1 & 4,19 & 1,13 & 3,06 \\ \hline
			2 & 4,23 & 1,02 & 3,21 \\ \hline
			3 & 4,22 & 1,14 & 3,08\\ \hline
			4 & 4,21 & 1,15 & 3,06\\ \hline
			5 & 4,17 & 1,17 & 3,00\\ \hline
			6 & 4,16 & 1,19 & 2,97\\ \hline
			7 & 4,16 & 1,17 & 2,99\\ \hline
			8 & 4,21 & 1,15 & 3,06\\ \hline
			9 & 4,19 & 1,17 & 3,02\\ \hline
			10 & 4,19 & 1,19 & 3,00\\ \hline
		\end{tabular}
	\end{table}
\begin{center}
	średnia grubość pozorna \textit{h} - \\
	niepewność \textit{u(h)} -
\end{center}
		\newpage
		\paragraph{Obliczenie grubości rzeczywistej dla płytki pleksiglasowej} WSTAWIĆ OBLICZENIA
		\\	\begin{table}[!htbp]
			\caption{\textbf{}}
			\centering
			\def\arraystretch{1.4}
			\begin{tabular}{|c|c|c|c|}
				\hline
				\multicolumn{4}{|l|}{\makecell{materiał: pleksiglas\hspace{95pt} \\ grubość rzeczywista d = WSTAWIĆ[mm] \\ niepewność \hspace{55pt}$u(d) = 0,01$[mm]\hspace{20pt}}} 
				\\ \hline
				\multirow{2}{*}{lp.} & \multicolumn{2}{c|}{Wskazanie czujnika} & \makecell{grubość \\pozorna } \\
				\cline{2-4}
				
				\multirow{2}{*}{} & $a_d$[mm] & $a_g$[mm] & $h = a_d - a_g$[mm] \\ \hline
				1 & 4,39 & 1,74 & 2,65 \\ \hline
				2 & 4,38 & 1,80 & 2,38 \\ \hline
				3 & 4,36 & 1,74 & 2,62\\ \hline
				4 & 4,35 & 1,79 & 2,56\\ \hline
				5 & 4,35 & 1,76 & 2,59\\ \hline
				6 & 4,42 & 1,82 & 2,60\\ \hline
				7 & 4,39 & 1,76 & 2,63\\ \hline
				8 & 4,38 & 1,79 & 2,59\\ \hline
				9 & 4,41 & 1,78 & 2,63\\ \hline
				10 & 4,33 & 1,78 & 2,55\\ \hline
		\end{tabular}
	\end{table}
	\begin{center}
		średnia grubość pozorna \textit{h} - \\
		niepewność \textit{u(h)} - 
	\end{center}
	\newpage
	
	\section{Opracowanie wyników pomiarów}
	\begin{enumerate}
		\item Oblicz moment bezwładności $I_0$ względem rzeczywistej osi obrotu korzystając z wzoru na okres drgań.
		\item Korzystając z twierdzenia Steinera oblicz moment bezwładności $I_S$ względem osi przechodzącej przez środek masy.
		\item Oblicz również moment bezwładności względem osi przechodzącej przez środek masy $I_S^{(geom)}$ na podstawie masy i wymiarów geometrycznych.
		\item Oblicz lub przyjmij niepewności wielkości mierzonych bezpośrednio: okresu $T$, masy $m$ i wymiarów geometrycznych.
		\item Oblicz niepewność złożoną momentu bezwładności $I_0$ oraz $I_S$.
		\item Obliczyć niepewność $u_c(I_S^{(geom)})$.
		\item Która z obydwu metod wyznaczania momentu bezwładności jest dokładniejsza?
		\item Czy w granicach niepewności rozszerzonej obydwa wyniki pomiaru są zgodne?
\end{enumerate}

\textit{Obliczenia dla pręta}
\noindent \\
ad 1: Przekształcając 
$$
T = 2\pi\sqrt{\frac{I_0}{mga}}
$$
otrzymujemy
$$
I_0 = \frac{mgaT^2}{4\pi^2} = \frac{663 * 10^{-3} * 9,81 * 274 * 10^{-3} * (1,32)^2}{4*3.14^2} = 0,0787[kg*m^2]
$$
ad 2: Z twierdzenia Steinera 
$$
I_S = I_0 - ma^2 = 0,0787 - (663*10^{-3}*(274*10^{-3})^2) = 0,0289[kg*m^2]
$$
ad 3: Z podręcznika odczytujemy $$I_S^{geom} = \frac{1}{12}ml^2 = \frac{1}{12}*663*10^{-3}*(748*10^{-3})^2 = 0,0309[kg*m^2]
$$
ad 4: Niepewności okresu (typu A): $$\bar{T} = \frac{\sum T_i}{n} \approx \frac{13,2}{10} = 1,32[s] ;\hspace{10pt} u(T) = \sqrt{\frac{\sum (T_i - \bar{T})^2}{n(n-1)}} = \sqrt{\frac{0,0004}{90}} = 0,0021[s] 
$$
\\ Niepewność masy: $u(m) = 1 g = 0,0010 kg$ 
\\ Niepewność długości pręta: $u(l) = 1 mm = 0,0010 m$
\\ Niepewność odległości $a = l/2 - b: u(a) = 0,5mm = 0,0005 m$
\\ad 5: Dla $I_0$ stosujemy prawo przenoszenia niepewności względnych i otrzymujemy 
$$
\frac{u(I_0)}{I_0} = \sqrt{\left[\frac{u(m)}{m}\right]^2 + \left[\frac{u(a)}{a}\right]^2 + \left[2\frac{u(T_0)}{T_0}\right]^2} = \sqrt{\left(\frac{0,0010}{0,663}\right)^2 + \left(\frac{0,0005}{0,274}\right)^2 + \left(2\frac{0,0021}{1,32}\right)^2} = 0,0040
$$
$$
u(I_0) = 0,0040 * 0,0787 = 0,0003[kg * m^2]
$$
Dla $I_S$ stosujemy zwykłe prawo przenoszenia niepewności i otrzymujemy 
$$
u(I_S) = \sqrt{[u(I_0)]^2 + [a^2 * u(m)]^2 + [-2 am * u(a)]^2} =$$ $$= \sqrt{0,0003^2 + (0,274^2 * 0,001)^2 + (-2*0,274*0,663*0,0005)^2} = $$
$$ = 0,0004[kg * m^2]
$$
ad 6: Z prawa przenoszenia niepewności względnych otrzymujemy:
$$
\frac{u(I_S^{(geom)})}{I_S^{(geom)}} = \sqrt{\left[\frac{u(m)}{m}\right]^2 + \left[2*\frac{u(l)}{l}\right]^2} = \sqrt{\left(\frac{0,001}{0,663}\right)^2 + \left(2\frac{0,001}{0,748}\right)^2} = 0,0031
$$
$$
u(I_S^{(geom)}) = 0,0031 * 0,0309 = 0,0001[kg*m^2]
$$
ad 7: Niepewność dla wzoru podręcznikowego jest mniejsza niż niepewność przy stosowaniu wzoru Steinera więc sposób obliczania momentu bezwładności korzystając z masy i wymiarów geometrycznych jest dokładniejszy.
\\ad 8: Obliczamy stosunek
$$
\frac{|I_S - I_S^{(geom)}|}{\sqrt{u^2(I_S) +  u^2(I_S^{(geom)})}} = \frac{|0,0289 - 0,0309|}{\sqrt{0,0004^2 + 0,0001^2}} = \frac{0,002}{0,0004} = 5
$$
Ponieważ współczynnik jest większy od 2 wyników nie można uznać za zgodne.
\vspace{20pt}
\noindent
\\Dla pierścienia
\noindent
ad 1: Przekształcając 
$$
T = 2\pi\sqrt{\frac{I_0}{mga}}
$$
otrzymujemy
$$
I_0 = \frac{mgaT^2}{4\pi^2} = \frac{1343 * 10^{-3} * 9,81 * 130 * 10^{-3} * (1,02)^2}{4*3.14^2} = 0,0452[kg*m^2]
$$
ad 2: Z twierdzenia Steinera 
$$
I_S = I_0 - ma^2 = 0,452 - (1343*10^{-3}*(130*10^{-3})^2) = 0,0225[kg*m^2]
$$
ad 3: Z podręcznika odczytujemy $$I_S^{geom} = \frac{1}{2}m(R_z^2+ R_w^2) = \frac{1}{2}*1343*10^{-3}*(0,140^2 + 0,126^2) = 0,0238[kg*m^2]
$$
ad 4: Niepewnośc okresu (typu A): $$\bar{T} = \frac{\sum T_i}{n} \approx \frac{10,2}{10} = 1,02[s] ;\hspace{10pt} u(T) = \sqrt{\frac{\sum (T_i - \bar{T})^2}{n(n-1)}} = \sqrt{\frac{0,0005}{90}} = 0,0024[s] 
$$
\\ Niepewność masy: $u(m) = 1 g = 0,0010 kg$ 
\\ Niepewność długości średnicy zewnętrznej: $u(R_z) = 1 mm = 0,0010 m$
\\ Niepewność długości średnicy wewnętrznej:
 $u(R_w) = 1 mm = 0,0010 m$
\\Niepewność pomiaru $e = 1mm = 0,0010$
\\ Niepewność długości $a = R_z/2 - e: u(a) = 0,5mm = 0,0005 m$
\\ad 5: Dla $I_0$ stosujemy prawo przenoszenia niepewności względnych i otrzymujemy 
$$
\frac{u(I_0)}{I_0} = \sqrt{\left[\frac{u(m)}{m}\right]^2 + \left[\frac{u(a)}{a}\right]^2 + \left[2\frac{u(T_0)}{T_0}\right]^2} = \sqrt{\left(\frac{0,0010}{1,343}\right)^2 + \left(\frac{0,0005}{0,130}\right)^2 + \left(2\frac{0,0024}{1,02}\right)^2} = 0,0061
$$
$$
u(I_0) = 0,0061 * 0,0452 = 0,0003[kg * m^2]
$$
Dla $I_S$ stosujemy zwykłe prawo przenoszenia niepewności i otrzymujemy 
$$
u(I_S) = \sqrt{[u(I_0)]^2 + [a^2 * u(m)]^2 + [-2 am * u(a)]^2} =$$ $$= \sqrt{0,0003^2 + (0,130^2 * 0,001)^2 + (-2*0,130*1,343*0,0005)^2} = $$
$$ = 0,0003[kg * m^2]
$$
ad 6: Z prawa przenoszenia niepewności względnych otrzymujemy:
$$
\frac{u(I_S^{(geom)})}{I_S^{(geom)}} = \sqrt{\left[\frac{u(m)}{m}\right]^2 + \left[2*\frac{u(R_z)}{R_z}\right]^2 + \left[2*\frac{u(R_w)}{R_w}\right]^2}=$$ $$ = \sqrt{\left(\frac{0,001}{1,343}\right)^2 + \left(2\frac{0,0005}{0,140}\right)^2 + \left(2\frac{0,0005}{0,126}\right)^2} = 0,0107$$
$$
u(I_S^{(geom)}) = 0,0107 * 0,0238 = 0,0003[kg*m^2]
$$
ad 7: Niepewność dla wzoru podręcznikowego i niepewność przy stosowaniu wzoru Steinera 
są takie same więc w tym wypadku obie metody były równie dokładne.
\\ad 8: Obliczamy stosunek
$$
\frac{|I_S - I_S^{(geom)}|}{\sqrt{u^2(I_S) +  u^2(I_S^{(geom)})}} = \frac{|0,0225 - 0,0238|}{\sqrt{0,0003^2 + 0,0003^2}} = \frac{0,0013}{0,0004} = 2,75
$$
Ponieważ współczynnik jest większy od 2 wyników nie można uznać za zgodne.

\begin{table}[h]
	\caption{Wyniki obliczeń momentów bezwładności dla pręta}
 	\begin{tabular}[hb]{|c|c|c|c|}
 		\hline
 		& $I_0$ z okresu drgań [$kg*m^2$] & $I_S$ z twierdzenia Steinera [$kg*m^2$]& $I_S^{(geom)} [kg*m^2]$ \\ \hline
 		Wartość&0,0787&0,0289&0,0309 \\ \hline
 		Niepewność&0,0003&0,0004&0,0001 \\ \hline
 	\end{tabular}
\end{table}

\begin{table}[h]
	\caption{Wyniki obliczeń momentów bezwładności dla pierścienia}
	\begin{tabular}[hb]{|c|c|c|c|}
		\hline
		& $I_0$ z okresu drgań [$kg*m^2$] & $I_S$ z twierdzenia Steinera [$kg*m^2$]& $I_S^{(geom)} [kg*m^2]$ \\ \hline
		Wartość&0,0452&0,0225&0,0238 \\ \hline
		Niepewność&0,0003&0,0003&0,0003 \\ \hline
	\end{tabular}
\end{table}
	\section{Wnioski}
	Moment bezwładności obliczony dla pręta z pomiaru czasu drgań wahadła wynosi $0,0787 \pm 0,0003  [kg\cdot m^2]$. Nie jest to wartość zgodna z wartością odczytaną z podręcznika, która wynosi $0,0289[kg*m^2]$. Zatem wyników nie możemy uznać za poprawne. \\
	Moment bezwładności obliczony dla pierścienia z pomiaru czasu drgań wahadła wynosi $0,0452 \pm 0,0003  [kg\cdot m^2]$. Nie jest to wartość zgodna z wartością odczytaną z podręcznika, która wynosi $0,0238[kg*m^2]$. Zatem wyników nie możemy uznać za poprawne.\\
	Pomiary były dokonywane ręcznie, a na poprawność wyników mógł wpłynąć czas reakcji człowieka, który decydował czy mierzony przyrząd jest w tej samej fazie z której rozpoczynany był pomiar. Dodatkowo określenie kąta wychylenia było subiektywne, a nie powinno przekraczać trzech stopni.
\end{document}
















\end{}