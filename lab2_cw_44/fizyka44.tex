\documentclass[a4paper,10pt,twoside]{article}
%\usepackage{amssymb}
%\usepackage{amsthm}
\usepackage[polish]{babel}
\usepackage[utf8]{inputenc}
\usepackage[T1]{fontenc}
\usepackage{indentfirst}
\usepackage{caption}
\usepackage[top=2.5cm, bottom=2.5cm, left=2.5cm, right=2.5cm]{geometry}
\usepackage{graphicx}
\usepackage{makecell}
\usepackage{amsmath}
\usepackage{booktabs}
\usepackage{multirow}
\usepackage{float}


\begin{document}
	
\begin{table}[]
	\centering
	\begin{tabular}{lllllll}
		\cline{1-6}
		\multicolumn{1}{|c|}{\begin{tabular}[c]{@{}c@{}}EAiIB\\ Informatyka\end{tabular}}              & \multicolumn{2}{l|}{\begin{tabular}[c]{@{}l@{}}~~~~~~~~~~~~~~~Michał Kilian\\ ~~~~~~~~~~~~~~~Mateusz Ficek\end{tabular}}                                                                                                & \multicolumn{1}{c|}{\begin{tabular}[c]{@{}c@{}}Rok\\ II\end{tabular}}          & \multicolumn{1}{c|}{\begin{tabular}[c]{@{}c@{}}Grupa\\ 5a\end{tabular}}            & \multicolumn{1}{c|}{\begin{tabular}[4c]{@{}c@{}}Zespół\\ 6\end{tabular}}      &  \\ \cline{1-6}
		\multicolumn{1}{|c|}{\begin{tabular}[c]{@{}c@{}}Pracownia\\ FIZYCZNA\\ WFiIS AGH\end{tabular}} & \multicolumn{4}{l|}{\begin{tabular}[c]{@{}l@{}}Temat:\\ \textbf{Indukcyjność wzajemna} \end{tabular}}                                                                                                                                                                                                                                            & \multicolumn{1}{l|}{\begin{tabular}[c]{@{}l@{}}nr ćwiczenia:\\ ~~~~~~~~44\end{tabular}} &  \\ \cline{1-6}
		\multicolumn{1}{|l|}{\begin{tabular}[c]{@{}c@{}}Data wykonania:\\ 22.10.2018\end{tabular}}      & \multicolumn{1}{c|}{\begin{tabular}[c]{@{}c@{}}Data oddania:\\ 26.10.2018\end{tabular}} & \multicolumn{1}{l|}{\begin{tabular}[c]{@{}l@{}}Zwrot do poprawki:\\ \phantom{data poprawki}\end{tabular}} & \multicolumn{1}{l|}{\begin{tabular}[c]{@{}l@{}}Data oddania:\\  \phantom{data oddania}\end{tabular}} & \multicolumn{1}{l|}{\begin{tabular}[c]{@{}l@{}}Data zaliczenia:\\  \phantom{data zaliczenia}\end{tabular}} & \multicolumn{1}{l|}{\begin{tabular}[c]{@{}l@{}}OCENA:\\ \phantom{ocena}\end{tabular}}       &  \\ \cline{1-6}
		&                                                                                         &                                                                                     &                                                                                &                                                                                   &                                                                               & 
	\end{tabular}
\end{table}



	\section{Cel ćwiczenia}
		Pomiar współczynnika indukcji wzajemnej dwóch cewek sprzężonych ze sobą magnetycznie, dla różnych położeń tych cewek.
	\section{Wprowadzenie}
	Dołączone w na osobnych kartkach
	\section{Wykonanie ćwiczenia}
	
	\begin{enumerate}
		\item Zestawić obwód pomiarowy
		\item Dokonać pomiaru indukcyjności wypadkowej dla dodatniego i ujemnego sprzężenia cewek powietrznych przy różnych odległościach cewek (odległości te zmieniać co 0,5 cm).
		\item Zmierzyć indukcyjność własną obu cewek.
		\item Wyniki notować w osobiście zaprojektowanej tabeli, zawierającej również rezultaty obliczeń $M$ oraz $k$.
	\end{enumerate}
\newpage
	
	\section{Wyniki pomiarów}
	
\begin{table}[htb]
	\begin{tabular}{|c|c|c|c|c|c|}
		\hline
		Indeks & $L_p$ {[}H{]} & $L_z$ {[}H{]} & odległość[cm] & Indukcyjność wzajemna $M${[}H{]} & Współczynnik sprzężenia $k$ \\ \hline
		1      & 2,69       & 3,67 & 0       & 0,25                          & 0,43                      \\ \hline
		2      & 2,69       & 3,65 & 0,5     & 0,24                          & 0,42                      \\ \hline
		3      & 2,7        & 3,63 & 1       & 0,23                          & 0,40                      \\ \hline
		4      & 2,71       & 3,61 & 1,5     & 0,23                          & 0,39                      \\ \hline
		5      & 2,72       & 3,59 & 2       & 0,22                          & 0,38                      \\ \hline
		6      & 2,74       & 3,58 & 2,5     & 0,21                          & 0,37                      \\ \hline
		7      & 2,76       & 3,55 & 3       & 0,20                          & 0,34                      \\ \hline
		8      & 2,78       & 3,53 & 3,5     & 0,19                          & 0,33                      \\ \hline
		9      & 2,81       & 3,5  & 4       & 0,17                          & 0,30                      \\ \hline
		10     & 2,83       & 3,46 & 4,5     & 0,16                          & 0,27                      \\ \hline
		11     & 2,86       & 3,43 & 5       & 0,14                          & 0,25                      \\ \hline
		12     & 2,88       & 3,41 & 5,5     & 0,13                          & 0,23                      \\ \hline
		13     & 2,91       & 3,37 & 6       & 0,12                          & 0,20                      \\ \hline
		14     & 2,94       & 3,35 & 6,5     & 0,10                          & 0,18                      \\ \hline
		15     & 2,96       & 3,32 & 7       & 0,09                          & 0,16                      \\ \hline
		16     & 2,99       & 3,29 & 7,5     & 0,08                          & 0,13                      \\ \hline
		17     & 3,01       & 3,27 & 8       & 0,07                          & 0,11                      \\ \hline
		18     & 3,03       & 3,25 & 8,5     & 0,06                          & 0,10                      \\ \hline
		19     & 3,05       & 3,23 & 9       & 0,05                          & 0,08                      \\ \hline
		20     & 3,06       & 3,22 & 9,5     & 0,04                          & 0,07                      \\ \hline
		21     & 3,08       & 3,21 & 10      & 0,03                          & 0,06                      \\ \hline
		22     & 3,09       & 3,19 & 10,5    & 0,03                          & 0,04                      \\ \hline
		23     & 3,11       & 3,18 & 11      & 0,02                          & 0,03                      \\ \hline
		24     & 3,11       & 3,17 & 11,5    & 0,02                          & 0,03                      \\ \hline
		25     & 3,11       & 3,17 & 12      & 0,02                          & 0,03                      \\ \hline
		26     & 3,12       & 3,16 & 12,5    & 0,01                          & 0,02                      \\ \hline
		27     & 3,12       & 3,16 & 13      & 0,01                          & 0,02                      \\ \hline
		28     & 3,12       & 3,16 & 13,5    & 0,01                          & 0,02                      \\ \hline
		29     & 3,12       & 3,16 & 14      & 0,01                          & 0,02                      \\ \hline
		30     & 3,12       & 3,16 & 14,5    & 0,01                          & 0,02                      \\ \hline
	\end{tabular}
\end{table}
\noindent
$L_p$ - Indukcyjność wypadkowa przy przeciwnym nawinięciu \\
$L_z$ - Indukcyjność wypadkowa przy zgodnym nawinięciu \\
Indukcyjność własna $L_1 = 3,00$\\
Indukcyjność własna $L_2 = 0,11$

 \newpage
	
	\section{Opracowanie wyników pomiarów}
	\begin{enumerate}
		\item \textit{Obliczyć
			współczynniki  indukcji  wzajemnej 
			M
			oraz  współczynnika  sprzężenia 
			k
			dla  każdego  położenia 
			cewek. }\vspace{10pt}
		\\ Współczynnik M indukcyjności wzajemnej liczony był ze wzoru 
		$$ M = \frac{L_z - L_p}{4}$$ natomiast współczynnik k ze wzoru 
		$$k = \frac{M}{\sqrt{L_1 L_2}} $$ Wyniki zostały zawarte w tabeli.
		\item \textit{ Dla  cewki  powietrznej  wykonać
		wykres  zależności  wypadkowej  indukcyjności  układu  (sprzężenie  dodatnie
		i ujemne) oraz współczynnika sprzężenia 
		k
		od odległości cewek}\vspace{10pt} \\Wykresy zostały zamieszczone poniżej
		\item \textit{Skomentować wyniki}
	\end{enumerate}
	\begin{figure}[H]
		\centering{\includegraphics[scale=0.6]{W1}}
	\end{figure}
		\begin{figure}[H]
		\centering{\includegraphics[scale=0.6]{W2}}
	\end{figure}
	\begin{figure}[H]
	\centering{\includegraphics[scale=0.6]{W3}}
	\end{figure}
 \newpage

	\section{Wnioski}

\end{document}
















\end{}